\section{Conclusion}

This paper broadly follows three parts: First a structural causal model defining the data generating process of the returns of stocks was formulated. Using daily returns for 11 stock for the last 30 years, relevant statistics was calculated, and used for tuning the parameters of the DGP. Using the structural causal model a data set was simulated. 2) An ensemble of three algorithms that predicts the Sharpe ratio of individual stocks was presented. Using these three models, I tune, train and select an algorithm on the simulated data set. I find the best performing algorithm being a model that estimates the sharp ratio for individual stocks by storing a rolling set of observations and calculating the relevant moments using this memory. 3) Using the tuned algorithm we apply it out-of-sample to the real data. Using the rolling Sharpe algorithm I find (comparing it to 4) other benchmarks, the algorithm performs extremely well, having a Sharpe ratio at $0.1256$ comparing to a tangency portfolio with a Sharpe ratio of $0.05545$. I discuss the different challenges of tuning an LSTM, and the value of using a structural causal model as test environment for trading algorithms. Finally I point out the challenges of using machine learning models in-sample, and express a word of caution of doing so.
