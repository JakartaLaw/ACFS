\section{Abstract}

NOTER:
- JEG MANGLER INTRODUKTION

- JEG MANGLER LITTERATUR REVIEW (SKAL VÆRE KORT) (GIV GERNE FORSLAG TIL GODE PAPIRER)

- JEG SKAL OGSÅ LIGE FINPUDSE KONKLUSIONEN

This paper investigates stock picking in an environment of structural breaks. A structural break, in this paper, is defined as an event that upends the data generating process of stock returns. I stride to find an algorithm that performs well at choosing the single stock with highest sharp ratio in each period.

I use data of 11 stocks for the past 30 years. First the data is presented, using graphs and summary statistics, for the individual stocks.

Next I present a structural causal model that attempts to model stock returns under a regime of structural breaks. I identify the parameters necessary for sampling from the structural model. Using the real data collected, I use log-likelihood estimation to find the parameter values. I simulate a data set of $11$ stocks with $2.000.000$ rows containing slightly less than $10.000$ structural breaks.

I move onto present three different algorithms for predicting the Sharpe ratio in a given time period. These being a deep learning model called LSTM, and two simpler algorithm using rolling estimations of Sharpe ratios. I discuss the different properties, and how each model should be trained and tuned.

Using the simulated data set i compare the three different algorithms and their capacity to accurately predict the Sharpe ratio of individual stocks. Using the best model i investigate the performance on the real data set comparing to different benchmarks. Using not only summary statistics of the different strategies for stock picking, but I also plot the counter factual investment strategies, and make a Monte Carlo simulation for comparing the performance of the different strategies. I find the best performing model to be the one using a rolling Sharpe ratio. Comparing to a tangency portfolio the Sharpe ratio is about twice as high: $0.12563$ compared to $0.05545$

In the end i discuss the challenges of tuning, training and selecting algorithms, and what this might entail for using in-sample-machine learning for developing trading algorithms. In more depth i discuss the challenges of using LSTM algorithms, and the limitations of my computer setup for efficiently training such an algorithm. Finally I discuss the value of creating controlled environments for trading algorithm development.
